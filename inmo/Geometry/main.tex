\documentclass[journal,12pt,onecolumn]{IEEEtran}
%
\usepackage{setspace}
\usepackage{gensymb}
%\doublespacing
\singlespacing

%\usepackage{graphicx}
%\usepackage{amssymb}
%\usepackage{relsize}
\usepackage[cmex10]{amsmath}
%\usepackage{amsthm}
%\interdisplaylinepenalty=2500
%\savesymbol{iint}
%\usepackage{txfonts}
%\restoresymbol{TXF}{iint}
%\usepackage{wasysym}
\usepackage{amsthm}
%\usepackage{iithtlc}
\usepackage{mathrsfs}
\usepackage{txfonts}
\usepackage{stfloats}
\usepackage{bm}
\usepackage{cite}
\usepackage{cases}
\usepackage{subfig}
%\usepackage{xtab}
\usepackage{longtable}
\usepackage{multirow}
%\usepackage{algorithm}
%\usepackage{algpseudocode}
\usepackage{enumitem}
\usepackage{mathtools}
\usepackage{tikz}
\usepackage{circuitikz}
\usepackage{verbatim}
%\usepackage{tfrupee}
\usepackage[breaklinks=true]{hyperref}
%\usepackage{stmaryrd}
\usepackage{tkz-euclide} % loads  TikZ and tkz-base
\usetkzobj{all}
\usepackage{listings}
    \usepackage{color}                                            %%
    \usepackage{array}                                            %%
    \usepackage{longtable}                                        %%
    \usepackage{calc}                                             %%
    \usepackage{multirow}                                         %%
    \usepackage{hhline}                                           %%
    \usepackage{ifthen}                                           %%
  %optionally (for landscape tables embedded in another document): %%
    \usepackage{lscape}     
\usepackage{multicol}
\usepackage{chngcntr}
%\usepackage{enumerate}

%\usepackage{wasysym}
%\newcounter{MYtempeqncnt}
\DeclareMathOperator*{\Res}{Res}
%\renewcommand{\baselinestretch}{2}
\renewcommand\thesection{\arabic{section}}
\renewcommand\thesubsection{\thesection.\arabic{subsection}}
\renewcommand\thesubsubsection{\thesubsection.\arabic{subsubsection}}

\renewcommand\thesectiondis{\arabic{section}}
\renewcommand\thesubsectiondis{\thesectiondis.\arabic{subsection}}
\renewcommand\thesubsubsectiondis{\thesubsectiondis.\arabic{subsubsection}}

% correct bad hyphenation here
\hyphenation{op-tical net-works semi-conduc-tor}
\def\inputGnumericTable{}                                 %%

\lstset{
%language=C,
frame=single, 
breaklines=true,
columns=fullflexible
}
%\lstset{
%language=tex,
%frame=single, 
%breaklines=true
%}

\begin{document}
%


\newtheorem{theorem}{Theorem}[section]
\newtheorem{problem}{Problem}
\newtheorem{proposition}{Proposition}[section]
\newtheorem{lemma}{Lemma}[section]
\newtheorem{corollary}[theorem]{Corollary}
\newtheorem{example}{Example}[section]
\newtheorem{definition}[problem]{Definition}
%\newtheorem{thm}{Theorem}[section] 
%\newtheorem{defn}[thm]{Definition}
%\newtheorem{algorithm}{Algorithm}[section]
%\newtheorem{cor}{Corollary}
\newcommand{\BEQA}{\begin{eqnarray}}
\newcommand{\EEQA}{\end{eqnarray}}
\newcommand{\define}{\stackrel{\triangle}{=}}

\bibliographystyle{IEEEtran}
%\bibliographystyle{ieeetr}


\providecommand{\mbf}{\mathbf}
\providecommand{\pr}[1]{\ensuremath{\Pr\left(#1\right)}}
\providecommand{\qfunc}[1]{\ensuremath{Q\left(#1\right)}}
\providecommand{\sbrak}[1]{\ensuremath{{}\left[#1\right]}}
\providecommand{\lsbrak}[1]{\ensuremath{{}\left[#1\right.}}
\providecommand{\rsbrak}[1]{\ensuremath{{}\left.#1\right]}}
\providecommand{\brak}[1]{\ensuremath{\left(#1\right)}}
\providecommand{\lbrak}[1]{\ensuremath{\left(#1\right.}}
\providecommand{\rbrak}[1]{\ensuremath{\left.#1\right)}}
\providecommand{\cbrak}[1]{\ensuremath{\left\{#1\right\}}}
\providecommand{\lcbrak}[1]{\ensuremath{\left\{#1\right.}}
\providecommand{\rcbrak}[1]{\ensuremath{\left.#1\right\}}}
\theoremstyle{remark}
\newtheorem{rem}{Remark}
\newcommand{\sgn}{\mathop{\mathrm{sgn}}}
\providecommand{\abs}[1]{\left\vert#1\right\vert}
\providecommand{\res}[1]{\Res\displaylimits_{#1}} 
\providecommand{\norm}[1]{\left\lVert#1\right\rVert}
%\providecommand{\norm}[1]{\lVert#1\rVert}
\providecommand{\mtx}[1]{\mathbf{#1}}
\providecommand{\mean}[1]{E\left[ #1 \right]}
\providecommand{\fourier}{\overset{\mathcal{F}}{ \rightleftharpoons}}
%\providecommand{\hilbert}{\overset{\mathcal{H}}{ \rightleftharpoons}}
\providecommand{\system}{\overset{\mathcal{H}}{ \longleftrightarrow}}
	%\newcommand{\solution}[2]{\textbf{Solution:}{#1}}
\newcommand{\solution}{\noindent \textbf{Solution: }}
\newcommand{\cosec}{\,\text{cosec}\,}
\providecommand{\dec}[2]{\ensuremath{\overset{#1}{\underset{#2}{\gtrless}}}}
\newcommand{\myvec}[1]{\ensuremath{\begin{pmatrix}#1\end{pmatrix}}}
\newcommand{\mydet}[1]{\ensuremath{\begin{vmatrix}#1\end{vmatrix}}}
%\numberwithin{equation}{section}
\numberwithin{equation}{subsection}
%\numberwithin{problem}{section}
%\numberwithin{definition}{section}
\makeatletter
\@addtoreset{figure}{problem}
\makeatother

\let\StandardTheFigure\thefigure
\let\vec\mathbf
%\renewcommand{\thefigure}{\theproblem.\arabic{figure}}
\renewcommand{\thefigure}{\theproblem}
%\setlist[enumerate,1]{before=\renewcommand\theequation{\theenumi.\arabic{equation}}
%\counterwithin{equation}{enumi}


%\renewcommand{\theequation}{\arabic{subsection}.\arabic{equation}}

\def\putbox#1#2#3{\makebox[0in][l]{\makebox[#1][l]{}\raisebox{\baselineskip}[0in][0in]{\raisebox{#2}[0in][0in]{#3}}}}
     \def\rightbox#1{\makebox[0in][r]{#1}}
     \def\centbox#1{\makebox[0in]{#1}}
     \def\topbox#1{\raisebox{-\baselineskip}[0in][0in]{#1}}
     \def\midbox#1{\raisebox{-0.5\baselineskip}[0in][0in]{#1}}

\vspace{3cm}

\title{
%	\logo{
Geometry: Maths Olympiad
%	}
}
\author{ G V V Sharma$^{*}$% <-this % stops a space
	\thanks{*The author is with the Department
		of Electrical Engineering, Indian Institute of Technology, Hyderabad
		502285 India e-mail:  gadepall@iith.ac.in. All content in this manual is released under GNU GPL.  Free and open source.}
	
}	
%\title{
%	\logo{Matrix Analysis through Octave}{\begin{center}\includegraphics[scale=.24]{tlc}\end{center}}{}{HAMDSP}
%}


% paper titles
% can use linebreaks \\ within to get better formatting as desired
%\title{Matrix Analysis through Octave}
%
%
% author names and IEEE memberships
% note positions of commas and nonbreaking spaces ( ~ ) LaTeX will not break
% a structure at a ~ so this keeps an author's name from being broken across
% two lines.
% use \thanks{} to gain access to the first footnote area
% a separate \thanks must be used for each paragraph as LaTeX2e's \thanks
% was not built to handle multiple paragraphs
%

%\author{<-this % stops a space
%\thanks{}}
%}
% note the % following the last \IEEEmembership and also \thanks - 
% these prevent an unwanted space from occurring between the last author name
% and the end of the author line. i.e., if you had this:
% 
% \author{....lastname \thanks{...} \thanks{...} }s
%                     ^------------^------------^----Do not want these spaces!
%
% a space would be appended to the last name and could cause every name on that
% line to be shifted left slightly. This is one of those "LaTeX things". For
% instance, "\textbf{A} \textbf{B}" will typeset as "A B" not "AB". To get
% "AB" then you have to do: "\textbf{A}\textbf{B}"
% \thanks is no different in this regard, so shield the last } of each \thanks
% that ends a line with a % and do not let a space in before the next \thanks.
% Spaces after \IEEEmembership other than the last one are OK (and needed) as
% you are supposed to have spaces between the names. For what it is worth,
% this is a minor point as most people would not even notice if the said evil
% space somehow managed to creep in.



% The paper headers
%\markboth{Journal of \LaTeX\ Class Files,~Vol.~6, No.~1, January~2007}%
%{Shell \MakeLowercase{\textit{et al.}}: Bare Demo of IEEEtran.cls for Journals}
% The only time the second header will appear i/year/1963s for the odd numbered pages
% after the title page when using the twoside option.
% s
% *** Note that you probably will NOT want to include the author's ***
% *** name in the headers of peer review papers.                   ***
% You can use \ifCLASSOPTIONpeerreview for conditional compilation here if
% you desire.




% If you want to put a publisher's ID mark on the page you can do it like
% this:
%\IEEEpubid{0000--0000/00\$00.00~\copyright~2007 IEEE}
% Remember, if you use this you must call \IEEEpubidadjcol in the second
% column for its text to clear the IEEEpubid ma/year/1963rk.



% make the title area
\maketitle



%\tableofcontents

\bigskip

\renewcommand{\thefigure}{\theenumi}
\renewcommand{\thetable}{\theenumi}
%\renewcommand{\theequation}{\theenumi}

%\begin{abstract}
%%\boldmath
%In this letter, an algorithm for evaluating the exact analytical bit error rate  (BER)  for the piecewise linear (PL) combiner for  multiple relays is presented. Previous results were available only for upto three relays. The algorithm is unique in the sense that  the actual mathematical expressions, that are prohibitively large, need not be explicitly obtained. The diversity gain due to multiple relays is shown through plots of the analytical BER, well supported by simulations. 
%
%\end{abstract}
% IEEEtran.cls defaults to using nonbold math in the Abstract.
% This preserves the distinction between vectors and scalars. However,
% if the journal you are submitting to favors bold math in the abstract,
% then you can use LaTeX's standard command \boldmath ast the very start
% of the abstract to achieve this. Many IEEE journals frown on math
% in the abstract anyway.

% Note that keywords are not normally used for peerreview papers.
%\begin{IEEEkeywords}
%Cooperative diversity, decode and forward, piecewise linear
%\end{IEEEkeywords}



% For peer review papers, you can put extra information on the cover
% page as needed:
% \ifCLASSOPTIONpeerreview
% \begin{center} \bfseries EDICS Category: 3-BBND \end{center}
% \fi
%
% For peerreview papers, this IEEEtran command inserts a page break and
% creates the second title. It will be ignored for othesr modes.
%\IEEEpeerreviewmaketitle


%Download python codes using 
%\begin{lstlisting}
%svn co https://github.com/gadepall/school/trunk/ncert/computation/codes
%\end{lstlisting}

\renewcommand{\theequation}{\theenumi}
\begin{enumerate}[label=\arabic*.,ref=\theenumi]
%\begin{enumerate}[label=\arabic*.,ref=\thesubsection.\theenumi]
\numberwithin{equation}{enumi}
\item For any natural number n, (n $\geq$ 3), let f(n) denote the number of non-congruent integer-sided triangles with perimeter n (e.g., f(3) = 1, f(4) = 0, f(7) = 2). Show that
\begin{enumerate}
\item $f(1999) > f(1996)$
\item f(2000) = f(1997).
\end{enumerate}









\item Let ABC be a triangle in which no angle is $90^{o}$. For any point P in the plane of the triangle, let $A_1, B_1, C_1$ denote the reflections of P in the sides BC, CA, AB respectively. Prove the following statements:
\begin{enumerate}
\item If P is the incentre or an excentre of ABC, then P is the circumcentre of $A_1B_1C_1$
\item If P is the circumcentre of ABC, then P is the orthocentre of $A_1B_1C_1$
\item If P is the orthocentre of ABC, then P is either the incentre or an excentre of
$A_1B_1C_1$.
\end{enumerate}

\item Let ABC be a triangle and D be the mid-point of side BC. Suppose $\angle$ DAB = $\angle$ BCA and $\angle$ DAC = $15^{o}$. Show that $\angle$ ADC is obtuse. Further, if O is the circumcentre of ADC, Prove that triangle AOD is equilateral.























\item For a convex hexagon ABCDEF in which each pair of opposite sides is unequal, consider the following statements:
\begin{enumerate}
\item $(a_1)$ AB is parallel DE    
\item $(a_2)$ AE = BD
\item $(a_1)$ BC is parallel EF      
\item $(a_2)$ BF = CE
\item $(a_1)$ CD is parallel FA      
\item $(a_2)$ CA = DF
\end{enumerate}

\begin{enumerate}
\item Show that if the all the six statements are true, then the hexagon is cyclic.
\item Prove that in fact, any five of these six statements also imply that the hexagon is cyclic. 
\end{enumerate}

























\item Consider an acute triangle ABC and let P be an interior point of ABC. Suppose the lines BP and CP, when produced, meet AC and AB in E and F respectively. Let D be the point where AP intersects the line segment EF and K be the foot of perpendicular from D on to BC. Show that DK bisects $\angle$EKF.

\item Let ABC be a triangle with sides a,b,c. Consider a triangle $A_1B_1C_1$ with sides equal to 
$a + \frac{b}{2}$, $b + \frac{c}{2}$, $c + \frac{a}{2}$. Show that
\begin{align*}
[A_1B_1C_1] \geq \frac{9}{4}[ABC],
\end{align*}
where [XYZ] denotes the area of the triangle XYZ.























\item Suppose p is a prime greater than 3. Find all pairs of integers (a,b) satisfying the equation
\begin{align*} 
a^{2} + 3ab + 2p(a + b) + p^{2} = 0
\end{align*}

\item If $\alpha$ is real root of the equation $x^{5} - x^{3} + x - 2 = 0$, prove that [$\alpha^{6}$] = 3.

\item Prove that the number of 5-tuples of positive integers (a,b,c,d,e) satisfying the equation
\begin{align*} 
abcde = 5(bcde + acde + abde + abce + abcd)
\end{align*}



















\item Let M be the midpoint of side BC of a triangle ABC. Let the median AM intersect ABC  at K and L,K being nearer to A than L. If AK = KL = LM, Prove that the sides of triangle ABC are in the ratio 5:10:13 in some order.

























\item Prove that for every positive integer n there exists a unique ordered pair (a,b) of positive integers such that
\begin{align*}
n = \frac{1}{2}(a + b - 1)(a + b - 2) + a.
\end{align*}

\item 
\begin{enumerate}
\item Prove that if n is a positive integer such that $n \geq 4011^{2}$, then there exists an integer l such that 
\begin{align*}
n < l^{2} < (1 + \frac{1}{2005})n.
\end{align*}
\item Find the smallest positive integer M for which whenever an integer n is such that $n \geq M$, there exists an integer l, such that 
\begin{align*}
n < l^{2} < (1 + \frac{1}{2005})n.
\end{align*}
\end{enumerate}



















\item Let $\sigma = (a_1, a_2, a_3,.....,a_n)$ be a permutation of (1, 2, 3,....,n). A pair $(a_i, a_j)$ is said to correspond to an inversion of $\sigma$, $i < j$ but $a_i > a_j$. (Example: In the permutation (2,4,5,3,1) there are 6 inversions corresponding to the pairs (2,1), (4,3),(4,1),(5,3),(5,1),(3,1).) How many permutations of (1,2,3,...,n), ($n \geq 3$), have exactly two inversions?
 







\item Define a sequence $\sum_{n=1}^{\infty}$ as follows:
\begin{align*}
a_n = \{0, if the number of positive divisors of n is odd,\\ 1,if the number of positive divisors of n is even \}
\end{align*}
Let $x=0.a_1a_2a_3$.... be the real number whose decimal expansion contains $a_n$ in the $n^{th}$ place $n \geq 1$. Determine with proof whether x is a rational or irrational.

\item Find all real numbers x such that 
\begin{align*}
[x^{2} + 2x] = [x^{2}] + 2[x]
\end{align*}

\item Let a,b,c be positive real numbers such that $a^{3} + b^{3} + c^{3}$. Prove that
\begin{align*}
a^{2} + b^{2} + c^{2} > 6(c - a)(c - b)
\end{align*}


















\item Call a natural number n faithful, if there exists natural numbers $a<b<c$ such that a divides b, b divides c and n = a + b + c.
\begin{enumerate}
\item Show that all but a finite number of natural numbers are faithful.
\item Find the sum of all natural numbers which are not faithful. 
\end{enumerate} 

\item Consider two polynomials 
\begin{align*}
P(x) = a_{n}x^{n} + a_{n-1}x^{n-1} + .....a_{1}x + a_{0} and\\
Q(x) = b_{n}x^{n} + b_{n-1}x^{n-1} + .....b_{1}x + b_{0}
\end{align*}
with integer coefficients such that $a_n - b_n$ is a prime, $a_{n-1} = b_{n-1}$ and $a_{n}b_{0} - a_{0}b_{n} \neq 0$. Suppose there exists a rational number r such that P(r) = Q(r) = 0. Prove that r is an integer.

















\item Let $p_1 < p_2 < p_3 < p_4$ and $q_1 < q_2 < q_3 < q_4$ be two sets of prime numbers such that $p_4 - p_1 = 8$ and $q_4 - q_1 = 8$. Suppose $p_1 > 5$ and $q_1 > 5$. Prove that 30 divides $p_1 - q_1$ .

















\item Let n be a positive integer. Call a nonempty subset S of $\{1,2,....n\}$ good if the arithmeteic mean of the elements of S is also an integer. Further let $t_n$ denote the number of good subsets $\{1,2,.....n\}$. Prove that $t_n$ and n are both odd or both even.





\item In an acute-angled triangle ABC, a point D lies on the segment BC. Let $O_1, O_2$ denote the circumcentres of triangles ABD and ACD, respectively. Prove that the line joining the circumcentre of triangle ABC and the orthocentre of triangle $O_{1}O_{2}D$ is parallel to BC.

\item In a triangle ABC, let D be a point on the segment BC such that AB + BD = AC + CD. Suppose that the
points B, C and the centroids of triangles ABD and ACD lie on a circle. Prove that AB = AC.
























\item From a set of 11 square integers, show that one can choose 6 numbers $a^{2}, b^{2}, c^{2}, d^{2}, e^{2}, f^{2}$ such that
\begin{align*}
 a^{2}+b^{2}+c^{2} = d^{2}+e^{2}+f^{2}.
\end{align*}

\item For any natural number $n > 1$, write the infinite decimal expansion of 1/n (for example, we write 1/2 = 0.49 its infinite decimal expansion, not 0.5). Determine the length of the non-periodic part of the (infinite) decimal expansion of $\frac{1}{n}$.














\item Let N denote the set of all natural numbers. Define a function T : N $\to$ N by
T(2k) = k and T(2k + 1) = 2k + 2. We write $T^{2}(n)$ = T(T(n)) and in general
$T^{k}(n) = T^{k-1}(T(n))$ for any $k > 1$.
\begin{enumerate}
\item Show that for each n $\in$ N, there exists k such that $T^{k}$(n) = 1.
\item For k $\in$ N, let $c_k$ denote the number of elements in the set $\{n : T^{k}(n) = 1\}$.
Prove that $c_{k+2} = c_{k+1} + c_{k}$s, for $k \geq 1$.
\end{enumerate}

\item Suppose 2016 points of the circumference of a circle are coloured red and the remaining points are coloured blue. Given any natural number n $\geq$ 3, Prove that there is a regular n-sided polygon all of whose vertices are blue.
\item Suppose n $\geq$ 0 is an integer and all the roots of $x^{3} + \alpha x + 4 - (2 \times 2016^{n})$ = 0
are integers. Find all possible values of $\alpha$.


\item Find the number of triples (x, a, b) where x is a real number and a, b belong to
the set \{1, 2, 3, 4, 5, 6, 7, 8, 9\} such that $x^{2} - a(x) + b = 0$,
where [x] denotes the fractional part of the real number x. (For example
(1.1) = 0.1 = (-0.9).)

\item Let n $\geq$ 1 be an integer and consider the sum
\begin{align*}
x = \sum_{k \geq 0} \begin{pmatrix} n \\ 2k \end{pmatrix} 2^{n-2k}3^{k} = \begin{pmatrix} n \\ 0 \end{pmatrix} 2^{n-2}.3 + \begin{pmatrix} n \\ 4 \end{pmatrix} 2^{n-4}3^{2} + .......
\end{align*}
Show that 2x - 1, 2x, 2x + 1 form the sides of a triangle whose area and inradius are also integers.













\end{enumerate}

\end{document}


